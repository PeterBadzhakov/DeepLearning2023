Genre classification based on book
covers with ConvNeXt
Introduction
This section discusses our approach to classifying
book genres using the ConvNeXt model, a cutting-
edge convolutional neural network (CNN). We lever-
age visual features from book covers and pre-trained
weights from ImageNet for this purpose.
Dataset
We used the Uchida Lab’s Book Dataset for Task
1, which contains approximately 50,000 book cover
images across 30 genres. The images were equally
split into training and test sets and resized to 224x224
pixels for uniformity.
Model Architecture
The ConvNeXt model, developed by Facebook AI
Research, served as the base of this architecture. The
model mixes the best aspects of Convolutional Net-
works and Transformers and performs well on image
tasks. It was initially frozen with ImageNet pre-
trained weights, and we introduced custom layers to
fine-tune it to our task.
Shortened Running Article Title
Figure 1: Accuracy
Training Process
Our model underwent two stages of training - initial
training and fine-tuning. In the initial stage, only
the custom layers were trained with the base model
layers remaining unchanged. If the validation loss
didn’t improve over three epochs, early stopping
was introduced to avoid overfitting. The last three
blocks of the base model were then unfrozen for
fine-tuning, with a reduced learning rate to preserve
useful features learned from ImageNet.
Results
Our ConvNeXt-based classifier was tested on unseen
data, achieving a test accuracy of 29.91% and a test
loss of 2.5330. However, improvements can still be
made.
Analysis
Our results indicate the model learned useful pat-
terns from the training data, but the complexity of
predicting a book’s genre based on its cover is still
a challenge. Factors such as varying cover designs
within a single genre add to this complexity. Fine-
tuning the last three blocks of the model, however,
has shown to be beneficial for performance.
Future Work
Future enhancements could come from adjusting the
model’s architecture, introducing data augmentation
techniques, or expanding the training data. As books
carry both textual and visual information, incorpo-
rating text-based features could boost the model’s
performance.
Conclusion
The success of this approach demonstrates the po-
tential of the ConvNeXt architecture for book genre
classification tasks. This endeavor forms part of a
larger mission to improve content discoverability and
further establish the usefulness of deep learning and
transfer learning in extracting visual cues for seman-
tic understanding.
Genre classification based on book
covers with ResNet
Introduction
This section focuses on another approach to classi-
fying book genres, this time employing the ResNet
model, a well-known and robust convolutional neural
network. The objective remains the same: leveraging
the visual cues on book covers to categorize them
into genres.
Dataset
Our dataset comprises 32,581 book cover images,
neatly distributed across 33 categories with roughly
1,000 images each. These images provide a diverse
visual representation of the book genres we aim to
classify.
Training Workflow
We begin by dividing the images into training and
testing sets, maintaining a 7:3 ratio. This gives us
22,806 images for training and 9,775 images for test-
ing. Each image in the dataset undergoes transfor-
mation, being resized to a 120x120 pixel square. For
validation purposes, we reserve 30% of the training
data. Nine images will be presented in a batch to pro-
vide a comprehensive view of the training process.
Model Architecture
The heart of our architecture is the ResNet-50 model.
ResNet’s core innovation is the introduction of "resid-
ual" or "skip" connections, which help the network
learn the residual mapping rather than the direct
underlying mapping. In simpler terms, instead of
learning the direct connection from input to output,
ResNet learns the difference (or "residual") between
the input and output, making the learning process
more effective and efficient.
Learning Rate
A crucial part of the training process is determining
an optimal learning rate, which is set at 0.01 based
on our analysis. We use a learning rate finder to
plot the loss function against various learning rates.
This visualization guides us towards a learning rate
2 Journal of Biological Sampling (2024) 12:533-684
Shortened Running Article Title
Figure 2: Suggested learning rate
that enables faster convergence and improved model
performance during training.
Results
Upon testing our ResNet-based classifier on unseen
data, we achieved an accuracy rate of 30%. While this
is a promising result, we are always looking for ways
to improve and refine our model’s performance.
Analysis
Our results indicate that the ResNet model is capable
of extracting useful patterns from our dataset, despite
the inherent complexity of book cover designs. The
residual learning strategy, which forms the core of
ResNet, seems to be an effective method for this task.
Future Work
Given these promising results, we see an opportunity
to fine-tune the model further. We could experi-
ment with various techniques such as adjusting the
model’s parameters, implementing additional image
transformations, or introducing more data augmen-
tation techniques. Overall, the application of the
ResNet model offers another fruitful direction for the
task of book genre classification from cover images.
Fiction detection from summary
Introduction
Initially the task at hand was to determine a book’s
genre based on book features (for example cover
description, blurb, other text on cover). This proved
harder than expected for a number of reasons:
1. Finding a suitable dataset is hard. Very few
datasets include the features required to build a
suitable model.
2. Genre is not an objective measure and most on-
line APIs and services rarely include genre la-
bels. Furthermore - when they happen to include
such labels - they differ between separate service
providers.
3. Extracting text from cover images introduces
significant information loss and noise. This is
due to the fact that most books from online cover
datasets and APIs often include irrelevant text
for the task at hand.
We have chosen to simplify the problem by reduc-
ing the number of classes to two.
The simplified problem is classifying a book as
fiction or non-fiction, based on book summary. The
CMU Book Summary dataset has been used for train-
ing and validation. The dataset has been prepro-
cessed - this includes label combination, class balanc-
ing and summary reduction. The model used is a
sequential neural network, featuring a bidirectional
LSTM layer for text processing. The trained model
displays a good ability to guess whether a book is
fiction or non-fiction, based on book summary.
Finding suitable data
To obtain suitable data, a number of methodologies
were employed. They include, but are not limited to:
1. Web scraping - this has proved futile as most
book information providers require some affilia-
tion with an institution. Our attempts to acquire
a higher quota on Google Book API were un-
successful. We abandoned this approach as we
had to piece together information from multiple
APIs and datasets.
2. We considered classifying books by their Dewey
Decimal Number. This is a standart classification
of books, but it differs from the original task,
because it is only loosely related to book genre.
DDC mostly focuses on book origin and subject
matter.
3. Using description from back cover (book blurb) -
this approach proved impossible as back covers
are rarely uploaded and often in bad quality.
Finally - we decided on using the CMU Book Sum-
mary Dataset. It features around 16,000 books. Each
book has a summary (around 500 words on average)
and one or multiple labels, related to book genre.
Creating a dataset - attempt 1
The CMU dataset (link in references below) is com-
prised of around 16000 rows each containing book
summary (which is around 500 words but varies),
and a JSON object containing book labels (genres).
Journal of Biological Sampling (2024) 12:533-684 3
Shortened Running Article Title
One book can have more than 1 class. There are a
total of 227 distinct labels, 200 of which appear less
than 100 times and are too few to be of interest. The
label distribution can be accesed through the follow-
ing link. The python script that summarizes dataset
label distribution can be found at the following link.
Firstly the following labels have been selected:
• "Gothic fiction",
• "Spy fiction",
• "Detective fiction",
• "Historical fiction",
• "Crime Fiction",
• "Science Fiction",
• "Speculative fiction",
• "Fiction"
The occurence rate of each label differs. The labels
are in ascending order with "Fiction" being the most
common one and "Gothic Fiction" being the least
common one.
A new dataset is constructed from the original
dataset. The JSON label column is parsed into a
more suitable one-hot encoded vector. If a label is
present for a given book - it is encoded as 1 at the
corresponding index in this one-hot encoded vector.
If a label is not present for a given book - it is encoded
as a 0 at the corresponding index. Such an encoding
for a book with labels "Spy fiction", "Detective fiction"
and "Fiction" is:
• "[0, 1, 1, 0, 0, 0, 0, 1]"
Also an "is-fiction" column is added to this new
dataset which contains 1 if for any book - any of the
selected labels are present.
This is the first dataset that has been constructed.
One big problem of this dataset is that it is highly
unbalanced with some labels occuring far more often
than others. Training a model with this dataset intro-
duces an obvious bias towards classes that are more
frequently represented in the dataset. This "faulty"
dataset can be found at the following link.
Creating a dataset - attempt 2
To address these issues - a second "balanced" dataset
has been constructed. The balanced dataset has
4000 entries of each kind - Fiction and Non-Fiction.
Whether an entry in the datset is fiction or not is
encoded with the "is-fiction" column. Also the one-
hot encoding of the different fiction labels has been
excluded from this balanced dataset.
Finally a big issue we faced was that the LSTM
layer of the model returned mostly the same result
- zero. This was due to the fact that the summaries
were too long and the signal died down upon pro-
cessing of more than 200 words. To address this - we
only included the first 100 words from each summary.
There are a few outliers with less than 100 words in
total for the whole summary but they do not com-
promise the overall performance of the model. This
dataset was used for the training of the final model.
The balanced dataset with short summaries can be
found at the following link.
The python script used to create the balanced
dataset can be found at the following link.
Validation dataset
A dataset used for validation has been constructed
from the books not features in the balanced dataset.
This dataset features short 100 word summaries of
books not featured in the original balanced dataset.
In total it has 471 entries and is only used for valida-
tion.
The python script used to create the balanced
dataset can be found at the following link.
Encoder layer
The first step of the classification process is defining
an encoder. A vocabulary consisting of the 10000
most frequent words is selected. An inner order of
words is established so each word correspongs to
some index in this inner order. If a word is not part
of the vocabulary a special index is selected for this
word - 1 corresponding to an unknown word ( in the
selected vocabulary).
An example of such an encoding:
1. Unencoded: Ash is an orphan foundling who
grew up as a camp follower with a group of
mercenaries.
2. Encoded: 6791 8 18 1115 1 19 1993 49 15 4 606
8303 11 4 151 3 4631
3. Encoded-decoded: ash is an orphan [UNK] who
grew up as a camp follower with a group of
mercenaries
Model architecture
The architecture of the final model is a sequential neu-
ral network, consisting of a several layers. The model
is implemented using the Tensorflow framework:
1. Encoder layer
2. Embedding layer - trainable embedding layer
with output dimension 100.
3. Bidirectional LSTM layer with 100 internal units.
We use L2 regularization and L2 recurrent regu-
larization with parameters 0.01.
4. Dense layer with 64 neurons, RELU activations
and L2 regularization with parameter 0.01. We
4 Journal of Biological Sampling (2024) 12:533-684
Shortened Running Article Title
apply dropout after the first dense layer to op-
timize the models generalization abilities and
prevent overfitting. Dropout probability is 0.5.
5. Second dense layer with 32 neurons, RELU ac-
tivations and L2 regularization with parameter
0.01. We also apply dropout after this layer with
dropout probability 0.5.
6. Final layer with 1 output neuron and RELU acti-
vation.
Training the model
During all of the training attempts we used the
ADAM optimizer. We experimented with multiple
loss function:
• MSE - Mean squared error proved quite unfruit-
ful because the required results are small num-
bers, most often in the range from 0 to 1. We
quickly opted for other loss function as MSE is
more usefull in regression tasks and we are faced
with a classification task.
• Binary Cross Entropy Loss - This loss proved
usefull in the training of out model. It produced
good results and this is what the final model has
been trained with.
The most useful metric used during training
proved to be binary accuracy. Our model produces
a number, most often in range from 0 to 1. Binary
accuracy takes the result from the model and inter-
prets the result as 1 if the model output is higher
than some threshold, otherwise interpets result as 0.
We found that the best results are achieved with a
threshhold of 0.6.
The model has been trained for around 30 epochs
with learning rate being decreased 10 times after each
10 epochs.
The final training loss is around 0.2134 and the
binary precision is around 95 percent on training
data and 90 percent on validation data.
The trained model can be downloaded from the
following link.
Failed attempts and architectures
We tried to create a model with several architectures
and datasets. Not all attempts produced good re-
sults. The following is a run-down of some of the
unsuccessful attempts to train a good model:
• Sigmoid activation function - using sigmoid acti-
vation function made quite a lot of sense in the
beginning but proved unfruitfull. The results
produced by the sigmoid activation function
were far worse than those when using RELU.
• Using full summaries - Initially we tried pass-
ing full summaries through the LSTM layer, but
we encountered a serious problem with that ap-
proach. When passing longer texts through the
LSTM layer we found out that the signal died
down and the LSTM layer always returned 0.
This unwanted behaviour could possibly be re-
moved with training the model with more data.
However, being limited by dataset size, we opted
to reduce the summary length and include only
the first 100 words from each summary.
• Multilabel classification - Multilabel classifica-
tion with the given dataset proved harder that
expected. The biggest problem we encountered
was that highly represented labels(classes) had
on average a higher output value that less rep-
resented ones. This could be solved by balanc-
ing the dataset, employing techniques of data
augmentation. One such technique could be
splitting each summary into 100 long word sum-
maries and including only as much summaries
from each desired class as for the dataset to be
balanced.
• Using standart recurrent layer (RNN layer) - We
found that a standart recurrent layer failed to
generalize the 100 word summaries and opted
to use an LSTM layer instead. Even for relatively
short 100 word summaries - the normal RNN
layer failed to give plausible results.